\documentclass{article}
\usepackage{color, colortbl}
\usepackage{amsmath}
\usepackage{graphicx}
\usepackage[a4paper, total={7in, 9in}]{geometry}
\definecolor{LightYellow}{rgb}{1,1,0.9}
\graphicspath{{images/}}

\title{CCE5225 \dashv{} Assignment 1 \\
\large MiniBooNE particle identification \\ signal/background classification}
\author{Sultan Dayani}
\date{December 2022}
\begin{document}

For all models I used the default parameters except for the ones stated in the text.
\section{Vanilla Neural Network}
\begin{tabular}{|c| c c|}
\hline
Hidden Layer Size & Fit Time & Score \\ [0.5ex] 
\hline
(10,) & 99.88s & 0.927113   \\
(20,) & 100.26s & 0.931188   \\
(30,) & 74.61s & 0.933052   \\
\rowcolor{LightYellow}
(40,) & 376.35s & 0.935090   \\
(50,) & 163.39s & 0.933744   \\ 
\hline
\hline
\rowcolor{LightYellow}
(40, 40) & 86.75s & 0.936723 \\
(40, 40, 40) & 163.92s & 0.934984 \\
(40, 40, 40, 40) & 249.26s & 0.932158 \\
\hline
\end{tabular}
% \includegraphics[width=0.6\textwidth]{1hiddenlayer_scores.png}

Using (40,40) and varying the activation function: \\
\begin{tabular}{|c| c c|}
	\hline  
	Activation & Fit Time & Score \\ [0.5ex] 
	\hline
	\rowcolor{LightYellow}
  'relu' & 86.75s & 0.936723 \\ 
  'logistic' & 168.04s & 0.930813 \\ 
  'tanh' & 164.20s & 0.937665 \\ 
  'identity' & 41.14s & 0.893321 \\ 
	\hline
\end{tabular} \\
The activation function 'tanh' gave the best results but takes twice the time 'relu' does.

Considering the time to train the model I choose a model with \textbf{hidden layer: (40,40), activation: 'relu'}.

\paragraph[Comment]{
The number of hidden layers helps a neural network to learn complex relationships, however too many hidden layers lead to overfitting which reduces performance.
It seems like two hidden layers with 40 neurons lead to a model that understands the relationship of the features well.
The activation functions lie close to each other, I believe the identity functions is the worst performing one as it is a linear function.
}
\section{SVM}
varying C on rbf: 
\begin{tabular}{|c | c c|}
\hline
C & Fit Time & Score \\ [0.5ex] 
\hline
0.1 & 331.12s & 0.877185 \\
1 & 160.40s & 0.891601 \\
\rowcolor{LightYellow}
10 & 148.25s & 0.906661 \\
\hline
\end{tabular}
% \includegraphics[width=0.6\textwidth]{svm_c_compiled.png}
\\ 
varying kernel with C: 10 
\begin{tabular}{|c | c c|}
	\hline
Kernel & Fit Time & Score \\ [0.5ex] 
	\hline
	\rowcolor{LightYellow}
  'rbf' & 148.25s & 0.906661 \\
  'linear' & 463.95s & 0.904653 \\
  'poly' & 274.41s & 0.832592 \\
  'sigmoid' & 393.91s & 0.728335 \\
	\hline
\end{tabular}
\\ 
The best parameters are: \textbf{'C': 10, 'kernel': 'rbf'} with a test score of 0.906661.

\paragraph[Comment]{
Regularization encourages the model to find a balance between complexity and accuracy which lead to a better performing model.
I assume in this example rbf lead to the best results because it does a better job capturing the non-linear relationship between the features and the target.
And also because 'rbf' is tuned by a single hyperparameter and I did not tune the other models hyperparameters.
}

% \pagebreak
\section{Random Forest Classifier}
\begin{tabular}{|c c c | c c|}
\hline
n-estimators & min-samples-split & max-depth & Fit Time & Test Score \\ [0.5ex] 
\hline
50 & 2 & 10 & 22.33s & 0.924643 \\
50 & 2 & 15 & 30.71s & 0.932408 \\
50 & 2 & 20 & 36.04s & 0.933984 \\ 
\hline
50 & 5 & 10 & 22.69s & 0.924575 \\
50 & 5 & 15 & 30.91s & 0.932860 \\
\rowcolor{LightYellow}
50 & 5 & 20 & 35.91s & 0.934388 \\ 
\hline
50 & 10 & 10 & 22.65s & 0.924864 \\
50 & 10 & 15 & 31.00s & 0.932552 \\
50 & 10 & 20 & 35.57s & 0.933110 \\ 
\hline
\hline
100 & 2 & 10 & 44.06s & 0.924883 \\
100 & 2 & 15 & 62.45s & 0.933225 \\
100 & 2 & 20 & 71.87s & 0.934695 \\ 
\hline
100 & 5 & 10 & 46.42s & 0.925114 \\
100 & 5 & 15 & 62.23s & 0.933292 \\
\rowcolor{LightYellow}
100 & 5 & 20 & 71.88s & 0.935205 \\ 
\hline
100 & 10 & 10 & 45.57s & 0.925162 \\
100 & 10 & 15 & 64.83s & 0.933273 \\
100 & 10 & 20 & 71.10s & 0.934484 \\
\hline
\end{tabular}

The best performing model is: \textbf{'n-estimators': 100, 'min-samples-split': 5, 'max-depth': 20} with a fit time of 71.88s and a test score of 0.935205.
\\ Considering the time to train the model a random forest with \textbf{n-estimators: 50, min-samples-split: 5, max-depth: 20} is a good performing one with 35sec fit time and 0.934 accuracy.

\paragraph[Comment]{
As seen from the data above the number of estimators or the min-samples-split does not have a significant impact on the performance of the model.
I believe max-depth has the biggest impact on the performance because it changes how complex of a pattern the model can learn.
}
\section{Conclusion}
% Vanilla Neural Network confusion matrix: \\ 
VNN-hyperparameters: \textbf{\{'hidden-layers': (40, 40), 'activation': 'relu'\}}  Accuracy: 0.939068 \\
% \begin{matrix}
% 	17765 & 893  \\
% 	684   & 6671 
% \end{matrix}
% SVM confusion matrix:
SVM-hyperparameters: \textbf{\{'C': 10, kernel='rbf'\}} Accuracy: 0.908468 \\
% \begin{matrix}
% 	17563 & 1095 \\
% 	1286  & 6069 
% \end{matrix}
% Random Forest confusion matrix:
RF hyperparameters: \textbf{\{'n-estimators': 50, 'min-samples-split': 5, 'max-depth': 20 \}} Accuracy: 0.935186 \\
% \begin{matrix}
% 	17852 & 806  \\
% 	880   & 6475 
% \end{matrix}
I believe SVM is the worst performing
model because it struggles to capture the complexity and non-linear relationship
 of the features. It is also the slowest model to train. \\  
\begin{figure}[hbp]
	\caption{Confusion Matrices from left to right: Vanilla Neural Network, SVM, Random Forest}
	\includegraphics[width=0.3\textwidth]{vnn_confusion_matrix.png}
	\includegraphics[width=0.3\textwidth]{svm_confusion_matrix.png}
	\includegraphics[width=0.3\textwidth]{rf_confusion_matrix.png}
\end{figure}
\end{document}